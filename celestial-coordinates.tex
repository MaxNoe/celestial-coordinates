% vim: set spell
% vim: spelllang=en
\PassOptionsToPackage{unicode}{hyperref}
\documentclass[aspectratio=1610, 9pt]{beamer}

\usetheme{vertex}


\setmathfont{XITS Math}[range={cal, bfcal}]

\usepackage[main=german, english]{babel}
\usepackage[autostyle]{csquotes}

\usepackage{xcolor}

\usepackage{tcolorbox}
\usepackage[]{siunitx}
\setmathfont{Fira Math}
\AtBeginDocument{\sisetup{math-rm=\mathrm,math-micro=μ}}

\usepackage{graphicx}
\usepackage{tikz}
\usepackage{calc}

% \usepackage{biblatex}

\usepackage{hyperref}
\usepackage{bookmark}

\usepackage{xparse}
\usepackage{expl3}

\RenewDocumentCommand \d {m} {\TextOrMath{\textd{#1}}{\mathinner{\symup{d}#1}}}


\title{Celestial Coordinates}%

\author[M. Nöthe]{Maximilian Nöthe}
\date[2020-10-09]{Group Seminar 2020-10-09}
\institute[E5b]{\includegraphics[height=0.1\textheight]{logos/e5b.pdf}}

\begin{document}

\maketitle

\begin{frame}[c]{Overview}
 \tableofcontents
\end{frame}

\begin{frame}{Where do I need to point my telescope}
\end{frame}

\section{Time}
\begin{frame}{The Two Concepts of Time}
  \begin{columns}[t, onlytextwidth]
    \begin{column}{0.475\textwidth}
      \begin{center}
        \textbf{\Large Earth's Orientation}\\[0.5\baselineskip]%
        \only<1>{%
          \includegraphics[height=3.5cm]{images/AxialTiltObliquity.png}\\[-1.2\baselineskip]
          \hspace{2.5cm}{\small[Dna-Dennis]}
        }%
        \only<2->{%
          \includegraphics[height=3.5cm]{images/vlba.jpeg}\\[-1.2\baselineskip]
          \hspace{-3.5cm}{\small[NASA]}
        }%
      \end{center}

      \begin{itemize}
        \item The historical concept
        \item Base of calendars $⇒ $ leap years
        \item Universal Time 1 (UT1)
      \end{itemize}
    \end{column}
    \begin{column}{0.525\textwidth}
      \onslide<3->{
        \begin{center}
          \textbf{\Large Linear Time}\\[0.5\baselineskip]%
          \includegraphics[height=3.5cm]{images/Atomuhr-CS2.jpg}\\[-1.2\baselineskip]
          \hspace{3.5cm}{\color{black!20}\small[Jörg Behrens]}
        \end{center}

        \begin{itemize}
          \item Monotonically ticking SI seconds
          \item Temps Atomique International (TAI)
        \end{itemize}
      }
    \end{column}
  \end{columns}
  \medskip
  \begin{columns}[t, onlytextwidth]
    \begin{column}{0.475\textwidth}
      \onslide<2->{%
        \begin{center}
          \large\bfseries
          Measured daily by radio telescopes\\ (e.\,g.\ by the VLBA)
        \end{center}
      }
    \end{column}
    \hfill
    \begin{column}{0.525\textwidth}
      \onslide<4->{%
        \begin{center}
          \large\bfseries
          Measured by >600 atomic clocks in 60 institutes\\
          (e.\,g.\ PTB Braunschweig)
        \end{center}
      }
    \end{column}
  \end{columns}
\end{frame}

\begin{frame}{Time Standards}
\end{frame}

\section{Terrestrial Coordinates and the Rotation of Earth}
\begin{frame}{title}
\end{frame}

\section{Precession / Nutation / Polar Motion}
\begin{frame}{title}
\end{frame}

\section{The Fundamental Celestial Reference System}
\begin{frame}{title}
\end{frame}

\section{Equatorial Coordinates}
\begin{frame}{title}
\end{frame}

\section{Horizontal Coordinates}
\begin{frame}{title}
\end{frame}

\section{Galactic Coordinates}
\begin{frame}{title}
\end{frame}

\section{Using \texttt{astropy} for Coordinate Transformation}
\bumper{Never do live demo}

\end{document}
